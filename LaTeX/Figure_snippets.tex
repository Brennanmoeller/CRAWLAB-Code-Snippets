% These all assume that you are using the suggested file structure:
%
% http://crawlab.org/wiki/index.php?title=LaTeX_Folders_and_Filenames
%
% This means that they expect your figures to be in a figures folder
% 
% All have a \vspace{-0.2in} after the figure. This may need to be removed 
% some templates.


% Standard Figure - Change width as necessary
%
\begin{figure}[tb]
\begin{center}
\includegraphics[width = 3in]{figures/figure_filename}
\caption{Caption}
\label{fig:descriptive_label}
\end{center}
\vspace{-0.2in}
\end{figure}
%


% Minipage environment to place figures side-by-side.
% Each figure gets its own caption.
%
\begin{figure}[tb]
\begin{center}
\begin{minipage}{0.45\textwidth}
\begin{center}
\includegraphics[width = \columnwidth]{figures/figure_filename_1}
\caption{Caption 1}
\label{fig:label_1}
\end{center}
\end{minipage}
\hspace{0.07\textwidth}
\begin{minipage}{0.45\textwidth}
\begin{center}
\includegraphics[width = \columnwidth]{figures/figure_filename_2}
\caption{Caption 2}
\label{fig:label_2}
\end{center}
\end{minipage}
\end{center}
\vspace{-0.2in}
\end{figure}
%


% The subfigure environment also places figures side-by-side or 
% one on top of the other, but give each subfigure a secondary label
% such as a. b. c. The entire group also gets a caption. This arrangement
% is best for figures that a very closely related.
%
% Only two subfigures are included in this snippet. Additional subplots
% could be added
% 
% To use, we need to include the subfigure package. It is included in the 
% CRAWLAB_OneColumn.tex template.
%
\usepackage{subfigure}

\begin{figure}[tb]
\begin{center}
\subfigure[Caption 1]
{
	\includegraphics[width=3in]{figures/figure_filename_1}
	\label{fig:label_1}
}
\subfigure[Caption 2]
{
	\includegraphics[width=3in]{figures/figure_filename_2}
	\label{fig:label_2}
}
\vspace{-0.125in}
\caption{Total Caption}
\label{fig:total_label}
\end{center}
\vspace{-0.2in}
\end{figure}
%